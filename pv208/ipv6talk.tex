% atol IPv6 presentation
% Martin Bukatovic, 2011
% www.fi.muni.cz/~xbukatov/atol

% Based on template by Sascha Frank, Sep. 2005 
% www.informatik.uni-freiburg.de/~frank/
% For non commercial purposes only.

\documentclass{beamer}

\usepackage[utf8]{inputenc}
\usepackage{beamerthemesplit}
\usepackage{listings}
\usepackage{url}

% theme
\usetheme{Boadilla}
\usecolortheme{crane}

\begin{document}
\lstset{basicstyle=\ttfamily}
\title[~]{IPv6} 
\author{Martin Bukatovič} 
\institute[PV208]{}
\date{16. 3. 2010}

\frame{\titlepage} 
\frame{\tableofcontents} 

%
% main
%

\section{Vlastnosti IPv6} 

\begin{frame} \frametitle{Vlastnosti IPv6}
IPv6 adresa zabírá 128\,b,
stroj má zpravidla několik různých adres pro jedno rozhraní.

% struktura adresy, co kdo prideluje a jak je to velke

Druhy adres:
\begin{itemize}
\item Unicast - individuální adresa
\item Anycast - výběrová adresa % smerovace, domaci agenti
\item Multicast - skupinová adresa % vsichni dle zony, solicited adresa uzlu ..
\end{itemize}

Adresní prostor:
\begin{itemize}
\item loopback: {\tt ::1/128}
\item lokální linkové adresy: {\tt fe80::/10}
\item unikátní lokální adresy: {\tt fc00::/7}
\item multicast: {\tt ff00::/8}
\item individuální globální adresy {\tt 2000::/3}
\end{itemize}

\end{frame}

%\begin{frame} \frametitle{Vlastnosti IPv6}
%Struktura globální individuální adresy:
%
%Adresy, co musi znat stroj
%Rozhrani muze mit nekolik adres
%Zona, dosah platnosti adresy
%
%\begin{block}{ukazka}
%%\begin{lstlisting}
%ping
%%\end{lstlisting}
%\end{block}
%
%\end{frame}

\section{Konfigurace IPv6} 

% Konfigurace sitovych parametru (prefix podsite, mtu, dns server, vychozi brana, ...) 

\begin{frame} \frametitle{Konfigurace IPv6}
Proces konfigurace může být:

\begin{itemize}
\item manuální - primárně pro servery
\item automatický - využívá zpráv protokolu ICMPv6
  \begin{itemize}
  \item bezestavová - stroj získá adresu přes oznámení směrovače
  \item DHCPv6 (stavová)
  \item bezestavové DHCPv6 - kombinace obou předchozích možností
  \end{itemize}
\end{itemize}

% ICMPv6 je pro fungování protokolu kritický a nelze jej jen tak beztrestně zakázat ...
% note: autokonf. jedine prirazuje adresu vychozi brany
%       dhcpv6 

\vskip 1em
\pause

Id rozhraní (posledních 64\,b adresy) si stroj generuje:
\begin{itemize}
\item sám z linkové adresy (dle EUI-64 pro IPv6)
\item vygeneruje náhodně (v případě privacy extensions)
\item generuje na základě kryptografického protokolu (při použití SEND)
\item nebo může převzít z DHCPv6 serveru % check
\end{itemize}

\end{frame}

\subsection{Autokonfigurace}
\begin{frame} \frametitle{Autokonfigurace IPv6}

Nový stroj v síti při bezestavové automatické konfiguraci:
\begin{enumerate}
\item vytvoří si lokální linkovou adresu
\item provede detekci duplicitních adres pro právě vytvořenou adresu, pokud selže, autokonfigurace končí
\item přidělí si lokální linkovou adresu
\item získá oznámení směrovače (obsahující všechny prefixy platné v podsíti)
\item přidělení adres ke všem prefixům s flagem A
\end{enumerate}

Konfigurace používající DHCPv6 se liší až v posledním kroku: místo automatického
přidělení adres z poskytnutých prefixů se kontaktuje DHCPv6 server.

% autokonfiguraci nelze vynechat, i kdyz pouzivame dhcp
% DHCPv6 předpokládá, že stroj má alespoň lokální linkovou adresu získanou autokonfigurací.
% Dle konfigurace uvedené v oznámení směrovače může stroj použít několik různých metod pro získání adresy.
% Pokud overeni adresy selze, autokonfigurace nepokracuje a nefunguje ani dhcp, ktere predpoklada aspon link-local adresu.
% pomoci dhcpv6 nepredam vychozi branu

\end{frame}

\subsection{DNS a IPv6}
\begin{frame} \frametitle{DNS a IPv6}
Původní návrh zprávy \texttt{oznámení směrovače} neobsahoval informace o DNS serverech. Vynucuje použití bezstavového DHCPv6.
% dhcp neumi rict def. branu a prefixy lokalni site, ale umi poskytnout adresu nebo dns server

RFC 6106 přidává do zprávy \texttt{oznámení směrovače} volby RDNSS (adresa DNS serveru) a DNSSL (domain search list), ale ještě nedávno experimentální, problém s implementacemi.

\vskip 1em
\pause

Ktere adresy by měly být zavedeny v DNS?
\begin{itemize}
\item všechny globální individuální adresy s dlouhodobou platností
\end{itemize}
Ale co adresy získané z bezstavové automatické konfigurace nebo DHCP? 

\end{frame}

\section{Překlad adres} 
\subsection{NAT a IPv4}
\begin{frame} \frametitle{NAT a IPv4}

Primární funkce překladu adres je vypořádat se s jejich nedostatkem.

\begin{block}{Network Address Translation}
Překlad adresy strojů z podsítě s neveřejným rozsahem IP adres (včetně portu) na omezeny počet veřejných adres (často jednu) a
umožňuje jim tak navázat spojeni směrem do Internetu.
\end{block}

Není možní se na stroje za NATem připojit z Internetu:
\begin{itemize}
\item \texttt{bad} - rozbíjí \texttt{end to end} komunikaci,
\item \texttt{ugly} - protokoly posílající adresy v datové části vyžadují zpracováni celého paketu (např. SIP, FTP),
\item \texttt{good} - přispívá určitou měrou k bezpečnosti podsítě (nejde ale o falešný pocit bezpečí?)
\end{itemize}

% bezpecnost zminit zde, takze se ji pak nebudu muset dale venovat
% viz: nektere protokoly prosakuji vnitrni ip adresu
%      pomoci smerovaci hlavicky je mozne projit natem (ale casto je zakazana)
%      bez fw nat bezpecny proste neni
%      + obrovsky rozsah ipv6 docela zneprijemnuje skenovani podsiti

IPv6 bylo navrženo tak, aby NAT nebyl potřeba, problem je, že NAT je dnes nasazen prakticky všude
a některé jeho vlastnosti jsou tak očekávány i od IPv6.
\end{frame}

\subsection{NAT v prostředí IPv6}
\begin{frame} \frametitle{Vedlejší funkce NATu a IPv6}

\begin{block}{Funkce}
Nezávislost adresace sítě na ISP, snadný přechod k jinému ISP a případně multihoming (připojení k několika ISP zároveň).
\end{block}

\begin{itemize}
\item Pro vnitřní komunikace používat výhradně unikátní lokální adresy (ULA), udržovat je ve vnitřním DNS serveru.
      Změna ISP není tak jednoduchá jako v případě NATu, je třeba přečíslovat globální adresy všech vnitřních uzlu (zjednodušené autokonfiguraci).
\pause
\item Použití PI adres.
      Používá se i v IPv4. Pro danou organizaci se vytvoří samostatný AS, který ma vlastni záznam v globálních směrovacích tabulkách a ISP ho jen směruje pres sebe.
      Vhodné pro multihoming velkých organizaci.
      Problematické z hlediska snahy udržovat pro IPv6 co nejmenší hierarchii.
% zminit multihoming malych organizaci - pres zmenu isp ...
% zadny navrh pro multihoming v ipv6 neni zatim hotovy

\end{itemize}
\end{frame}

\begin{frame} \frametitle{Vedlejší funkce NATu a IPv6}

\begin{block}{Funkce}
Nezávislost adresace sítě na ISP, snadný přechod k jinému ISP a případně multihoming (připojení k několika ISP zároveň).
\end{block}

Mapování prefixu sítí (NPTv6, původně nazývané NAT66):
% vychazi z nazvu NAT64 pro preklad mezi ipv4 a ipv6
\begin{itemize}
\item Podoba se NATu, překládá se jen prefix adresy.
\item Je možné se připojit do vnitřní sítě.
\item Jednoduché na konfiguraci i pro přechod k jinému ISP.  
\item Existuje experimentální podpora pro iptables.
% pretrvava ugly problem: posilani adres v datech
\end{itemize}

\end{frame}

\begin{frame} \frametitle{Vedlejší funkce NATu a IPv6}

\begin{block}{Funkce}
Zajištění soukromí a skrytí topologie vnitřní site před světem.
\end{block}

Řeší tzv. Privacy Extensions:
\begin{itemize}
\item jeden hlavni identifikátor rozhraní dle EUI-64, zavedeny v DNS pro nova příchozí spojeni na stroj
\item pro spojeni ven se periodicky generuji náhodné IPv6 adresy, které:
  \begin{itemize}
  \item jsou časově omezené,
  \item nejsou vedeny v DNS,
  \item pro spojeni ze stroje někam pryč mají přednost před primární adresou.
  \end{itemize}
% v jednom okamziku tak stroj muze pouzivat nekolik nahodnych ipv6 adres, ale pro nove spojeni pouziva vzdy tu nejnovejsi
% kterou po case pregeneruje (obvykle tak 1 den)
% otazka je, zda zpusobi vetsi chaos ipv6 nat nebo tohlecto
% viz. problem sledovani site, nahodne adresy nemusi byt unikatni ...
\end{itemize}

Problem s reverzním záznamem a potenciálním chaosem v síti.

\end{frame}

%
% reference
%

\section{Reference}

\begin{frame}
\frametitle{Reference}
\begin{itemize}

\item Pavel Satrapa: IPv6, \url{http://knihy.nic.cz/ipv6/}
\item lupa.cz, \url{http://www.lupa.cz/n/ipv6/clanky/}

\end{itemize}
\end{frame}

\end{document}
